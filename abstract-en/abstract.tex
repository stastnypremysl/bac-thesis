\documentclass[a4paper]{article}

\let\pdfcreationdate=\creationdate
\usepackage[a-2u]{pdfx}

\usepackage[utf8]{inputenc}
\usepackage{lmodern}

\hypersetup{
	unicode,
	pdfauthor={Přemysl Šťastný},
	pdftitle={Command-line tool lsql-csv for CSV files processing - abstract},
	pdfsubject={Command-line tool lsql-csv for CSV files processing - abstract},
	pdfkeywords={abstract, relational database, CSV, SQL, Haskell, query language, Unix philosophy},
	pdfproducer={LaTeX},
	pdfcreator={xelatex}
}

\begin{document}
lsql-csv is a tool for small CSV file data querying from a shell with short
  queries. It makes it possible to work with small CSV files like with a read-only
  relational databases. The tool implements a new language LSQL similar to SQL,
  specifically designed for working with CSV files in a shell. 
  LSQL aims to be a more lapidary language than SQL.
  Its design purpose is to enable its user to quickly write simple queries directly
  to the terminal.
\end{document}
