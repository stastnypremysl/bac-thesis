\chapter{User documentation}
\icode{lsql-csv} is a tool for CSV file data querying from a shell with short queries. It makes it possible to work with small CSV files like with a read-only relational database.

The tool implements a new language LSQL similar to SQL, specifically designed for working with CSV files in a shell. LSQL aims to be a more lapidary language than SQL. Its design purpose is to enable its user to quickly write simple queries directly
  to the terminal---its design purpose is therefore different from SQL, where the readability of queries is more taken into account than in LSQL.

\section{Installation}
It is necessary, you had \icode{GHC} ($\geq 8 <9.29$) and Haskell packages \icode{Parsec} ($\geq 3.1$ $<3.2$), \icode{Glob} ($\geq 0.10$ $<0.11$), 
\icode{base} ($\geq 4.9$ $<4.20$), \icode{text} ($\geq 1.2$ $<2.2$) and \icode{containers} ($\geq 0.5$ $<0.8$)
 installed. (The package boundaries given are identical to the boundaries in Cabal package file.) For a build and an installation run:

\begin{verbatim}
make
sudo make install
\end{verbatim}

Now the \icode{lsql-csv} is installed in \icode{/usr/local/bin}. If you want, you can specify \icode{INSTALL\_DIR} like:
\begin{verbatim}
sudo make INSTALL_DIR=/custom/install-folder install
\end{verbatim}
This will install the package into \icode{INSTALL\_DIR}.

If you have installed \icode{cabal}, you can alternatively run:
\begin{verbatim}
cabal install
\end{verbatim}
It will also install the Haskell package dependencies for you.

The package is also published at \url{https://hackage.haskell.org/package/lsql-csv} in the Hackage public repository. You can therefore also install it directly without the repository cloned with:
\begin{verbatim}
cabal install lsql-csv
\end{verbatim}

\subsection{Running the unit tests}
If you want to verify, that the package has been compiled correctly, it is possible to test it by running:
\begin{verbatim}
make test
\end{verbatim}
This will run all tests for you.

\section{\icode{lsql-csv}---quick introduction}

\subsection{Examples}
One way to learn a new programming language is by understanding concrete examples of its usage. The following examples are written explicitly for the purpose of teaching a reader, how to use the tool \icode{lsql-csv} by showing him many examples of its usage.

The following examples might not be enough for readers, who don't know enough Unix/Linux scripting. If this is the case, please consider learning Unix/\allowbreak Linux scripting first before LSQL.

It is also advantageous to know SQL.

The following examples will be mainly about parsing of \icode{/etc/passwd} and parsing of \icode{/etc/group}. To make example reading more comfortable, we have added \icode{/etc/passwd} and \icode{/etc/group} column descriptions from man pages to the text.

File \icode{/etc/passwd} has the following columns \cite{passwd}:
\begin{enumerate}
    \item login name;
    \item optional encrypted password;
    \item numerical user ID;
    \item numerical group ID;    
    \item user name or comment field;
    \item user home directory;
    \item optional user command interpreter.
\end{enumerate}
File \icode{/etc/group} has the following columns \cite{group}:
\begin{enumerate}
    \item group name;
    \item password;
    \item numerical group ID;
    \item user list separated by commas.
\end{enumerate}


\subsubsection{Hello World}
\begin{verbatim}
lsql-csv '-, &1.2 &1.1'
\end{verbatim}
This will print the second (\icode{\&1.2}) and the first column (\icode{\&1.1}) of a CSV file on the standard input. If you know SQL, you can read it like \icode{SELECT S.second, S.first FROM stdio S;}.

Commands are split by commas into blocks. The first block is (and always is) the from block. There are file names or \icode{-} (the standard input) separated by whitespaces. The second block in the example is the select block, also separated by whitespaces.

For example:
\begin{verbatim}
lsql-csv '-, &1.2 &1.1' <<- EOF
World,Hello
EOF
\end{verbatim}
It returns:
\begin{verbatim}
Hello,World
\end{verbatim}


\subsubsection{Simple filtering}

\begin{verbatim}
lsql-csv -d: '-, &1.*, if &1.3 >= 1000' </etc/passwd
\end{verbatim}
This will print lines of users whose \icode{UID} $\geq 1000$. It can also be written as:
\begin{verbatim}
lsql-csv -d: 'p=/etc/passwd, p.*, if p.3 >= 1000'
    
lsql-csv -d: 'p=/etc/passwd, &1.*, if &1.3 >= 1000'

lsql-csv -d: '/etc/passwd, &1.*, if &1.3 >= 1000'
\end{verbatim}
The \icode{-d:} optional argument means the primary delimiter is \icode{:}. In the few examples we used overnaming, which allows us to give a data source file \icode{/etc/passwd} a name \icode{p}.

If you know SQL, you can read it as \icode{SELECT * FROM /etc/passwd P WHERE P.UID \textgreater= 1000}. As you can see, the LSQL style is more compressed than standard SQL.

The output might be:
\begin{verbatim}
nobody:x:65534:65534:nobody:/var/empty:/bin/false
me:x:1000:1000::/home/me:/bin/bash
\end{verbatim}

If you specify delimiter specifically for \icode{/etc/passwd}, the output will be a comma delimited.
\begin{verbatim}
lsql-csv '/etc/passwd -d:, &1.*, if &1.3 >= 1000'
\end{verbatim}
It might return:
\begin{verbatim}
nobody,x,65534,65534,nobody,/var/empty,/bin/false
me,x,1000,1000,,/home/me,/bin/bash
\end{verbatim}

This happens because the default global delimiter, which is used for the output generation, is a comma.
The global delimiter changes by usage of the command-line optional argument, but remains unchanged by the usage of the attribute inside the from block.


\subsubsection{Named columns}
Let's suppose we have a file \icode{people.csv}:
\begin{verbatim}
name,age                                                                                                                                                                                                                                   
Adam,21                                                                                                                                                                                                                                    
Petra,23                                                                                                                                                                                                                                   
Karel,25
\end{verbatim}

Now, let's get all the names of people in \icode{people.csv} using the \icode{-n} named optional argument:
\begin{verbatim}
lsql-csv -n 'people.csv, &1.name'
\end{verbatim}
The output will be:
\begin{verbatim}
Adam
Petra
Karel
\end{verbatim}

As you can see, we can reference named columns by the name. The named optional argument \icode{-n} enables first-line headers.
If first-line headers are enabled by the argument, each column has two names under \icode{\&X}---the number name \icode{\&X.Y} and the actual name \icode{\&X.NAME}.

Now, we can select all columns with a wildcard \icode{\&1.*}:
\begin{verbatim}
lsql-csv -n 'people.csv, &1.*'
\end{verbatim}
As the output, we get:
\begin{verbatim}
Adam,21,21,Adam
Petra,23,23,Petra
Karel,25,25,Karel
\end{verbatim}

The output contains each column twice because the wildcard \icode{\&1.*} was evaluated to \icode{\&1.1, \&1.2, \&1.age, \&1.name}.
How to fix it?
\begin{verbatim}
lsql-csv -n 'people.csv, &1.[1-9]*'
\end{verbatim}
The output is now:
\begin{verbatim}
Adam,21
Petra,23
Karel,25
\end{verbatim}

The command can also be written as
\begin{verbatim}
lsql-csv -n 'people.csv, &1.{1,2}'
lsql-csv -n 'people.csv, &1.{1..2}'
lsql-csv 'people.csv -n, &1.{1..2}'
\end{verbatim}
The output will be in all cases still the same.


\subsubsection{Simple join}
Let's say, I am interested in the default group names of users. We need to join tables \icode{/etc/passwd} and \icode{/etc/group}. Let's do it.
\begin{verbatim}
lsql-csv -d: '/etc/{passwd,group}, &1.1 &2.1, if &1.4 == &2.3'
\end{verbatim}
What does \icode{/etc/\{passwd,group\}} mean? Basically, there are three types of expressions. The select, the from, and the arithmetic expression. 
In all select and from expressions, you can use the curly expansion and wildcards just like in \icode{bash} \cite{bash-reference-manual}.

Finally, the output can be something like this:
\begin{verbatim}
root:root
bin:bin
daemon:daemon
me:me
\end{verbatim}
The first column is the name of a user and the second column is the name of its default group.


\subsubsection{Basic grouping}
Let's say, I want to count users using the same shell.
\begin{verbatim}
lsql-csv -d: 'p=/etc/passwd, p.7 count(p.3), by p.7'
\end{verbatim}
And the output?
\begin{verbatim}
/bin/bash:7
/bin/false:7
/bin/sh:1
/bin/sync:1
/sbin/halt:1
/sbin/nologin:46
/sbin/shutdown:1
\end{verbatim}

You can see here the first usage of the by block, which is equivalent to \icode{group by} in SQL.

\subsubsection{Basic sorting}
Let's say, you want to sort your users by \icode{UID} with \icode{UID} greater than or equal to 1000 ascendingly.
\begin{verbatim}
lsql-csv -d: '/etc/passwd, &1.*, if &1.3 >= 1000, sort &1.3'
\end{verbatim}
The output might look like:
\begin{verbatim}
me1:x:1000:1000::/home/me1:/bin/bash
me2:x:1001:1001::/home/me2:/bin/bash
me3:x:1002:1002::/home/me3:/bin/bash
nobody:x:65534:65534:nobody:/var/empty:/bin/false
\end{verbatim}

The sort block is the equivalent of \icode{order by} in SQL.

If we wanted descendingly sorted output, we might create a pipe to the \icode{tac} command---the \icode{tac} command prints the lines in reverse order:
\begin{verbatim}
lsql-csv -d: '/etc/passwd, &1.*, if &1.3 >= 1000, sort &1.3' | tac
\end{verbatim}

\subsubsection{About nice outputs}
There is a trick, how to concatenate two values in the select expression: Write them without space.

But how will the interpreter know the ends of the values in a command? If the interpreter sees a char, that can't be part of the currently parsed value, it tries to parse it as a new value concatenated to the current one.
You can use quotes for it---quotes themselves can't be part of most value types like the column name or numerical constant.

As an example, let's try to format our basic grouping example.

\begin{verbatim}
lsql-csv -d: 'p=/etc/passwd, 
  "The number of users of "p.7" is "count(p.3)".", by p.7'
\end{verbatim}
The output might be:
\begin{verbatim}
The number of users of /bin/bash is 7.
The number of users of /bin/false is 7.
The number of users of /bin/sh is 1.
The number of users of /bin/sync is 1.
The number of users of /sbin/halt is 1.
The number of users of /sbin/nologin is 46.
The number of users of /sbin/shutdown is 1.
\end{verbatim}

As you can see, string formatting is sometimes very simple with LSQL.

\subsubsection{Arithmetic expression}
So far, we just met all kinds of blocks, and only the if block accepts the arithmetic expression, and the other accepts the select expression.
What if we needed to run the arithmetic expression inside the select expression? There is a special syntax \icode{\$(\ldots{})} for it.

For example:
\begin{verbatim}
lsql-csv -d: '/etc/passwd, $(sin(&1.3)^2 + cos(&1.3)^2)'
\end{verbatim}

It returns something like:
\begin{verbatim}
1.0
1.0
1.0
0.9999999999999999
...
1.0
\end{verbatim}

If we run:
\begin{verbatim}
lsql-csv -d: '/etc/passwd, $(&1.3 >= 1000), sort $(&1.3 >= 1000)'
\end{verbatim}

We get something like:
\begin{verbatim}
false
false
...
false
true
true
...
true
\end{verbatim}

\subsubsection{More complicated join}
Let's see more complicated examples.
\begin{verbatim}
lsql-csv -d: 'p=/etc/passwd g=/etc/group, p.1 g.1, if p.1 in g.4'
\end{verbatim}
This will print all pairs of users and its group excluding the default group. 
If you know SQL, you can read it as \icode{SELECT P.1, G.1 FROM /etc/passwd P, /etc/group G WHERE G.4 LIKE '\%' + P.1 + '\%' } with operator \icode{LIKE} case-sensitive and columns named by their column number.

How does \icode{in} work? It's one of the basic string level ``consist''. If A is a substring of B, then \icode{A in B} is \icode{true}. Otherwise, it is \icode{false}.

And the output?
\begin{verbatim}
root:root                                                                                                                                                                                                                                                                 
root:wheel                                                                                                                                                                                                                                                                
root:floppy                                                                                                                                                                                                                                                               
root:tape                                                                                                                                                                                                                                                                 
lp:lp                                                                                                                                                                                                                                                                     
halt:root                                                                                                                                                                                                                                                                 
halt:wheel 
\end{verbatim}

The example will work under the condition, that there isn't any username, which is an infix of any other username.

\subsubsection{More complicated\ldots{}}
The previous example doesn't give a very readable output. We can use \icode{group by} to improve it (shortened as \icode{by}).
\begin{verbatim}
lsql-csv -d: 'p=/etc/passwd g=/etc/group, 
  p.1 cat(g.1","), if p.1 in g.4, by p.1'
\end{verbatim}
The output will be something like:
\begin{verbatim}
adm:adm,disk,sys,
bin:bin,daemon,sys,
daemon:adm,bin,daemon,
lp:lp,
mythtv:audio,cdrom,tty,video,
news:news,
\end{verbatim}
It groups all non-default groups of a user to a one line and concatenates it delimited by \icode{,}.

How can we add default groups too?
\begin{verbatim}
lsql-csv -d: 'p=/etc/passwd g=/etc/group, 
  p.1 cat(g.1","), if p.1 in g.4, by p.1' |
lsql-csv -d: '- /etc/passwd /etc/group, 
  &1.1 &1.2""&3.1, if &1.1 == &2.1 && &2.4 == &3.3'
\end{verbatim}
This will output something like:
\begin{verbatim}
adm:adm,disk,sys,adm
bin:bin,daemon,sys,bin
daemon:adm,bin,daemon,daemon
lp:lp,lp
mythtv:audio,cdrom,tty,video,mythtv
news:news,news
\end{verbatim}

The first part of the command is the same as in the previous example. The second part inner joins the output
of the first part with \icode{/etc/passwd} on the username and \icode{/etc/group} on the default GID number and prints
the output of the first part with an added default group name.

The examples will also work under the condition, that there isn’t any username,
which is an infix of any other username.

\section{Usage}
Now, if you understand the examples, it is time to move forward to a more abstract description of the language and tool usage.

\subsection{Options}
\begin{verbatim}
-h
--help
\end{verbatim}
Shows a short command line help and exits before doing anything else.

\begin{verbatim}
-n
--named
\end{verbatim}
Enables the first-line naming convention in CSV files---first-line with column names. 

This works only on input files. 
Output is always without first-line column names.

\begin{verbatim}
-dCHAR
--delimiter=CHAR
\end{verbatim}
Changes the default primary delimiter. The default value is \icode{,}.

\begin{verbatim}
-sCHAR
--secondary-delimiter=CHAR
\end{verbatim}
Changes the default quote char (secondary delimiter). The default value is \icode{"}.

\subsection{Datatypes}
There are 4 datatypes considered: \icode{Bool}, \icode{Int}, \icode{Double}, and \icode{String}. 
\icode{Bool} is either \icode{true}/\icode{false}, \icode{Int} is at least a 30-bit integer, \icode{Double} is a double-precision floating point number, and \icode{String} is an ordinary char string.

During CSV data parsing, the following logic of datatype selection is used:
\begin{itemize}
    \item \icode{Bool}, if \icode{true} or \icode{false};
    \item \icode{Int}, if the POSIX ERE \icode{[0-9]+} fully matches;
    \item \icode{Double}, if the POSIX ERE \icode{[0-9]+\textbackslash{}.[0-9]+(e[0-9]+)?} fully matches;
    \item \icode{String}, if none of the above matches.
\end{itemize}

\subsection{Joins}
Join means, that you put multiple input files into the from block.

Joins always have the time complexity $\mathcal{O}(nm)$. 
There is no optimization made based on if conditions when you put multiple files into the from block.

\subsection{Documentation of language}
\begin{verbatim}
lsql-csv [OPTIONS] COMMAND
      
Description of the grammar:
  
  COMMAND -> FROM_BLOCK, REST

                            
  REST -> SELECT_BLOCK, REST 
  REST -> BY_BLOCK, REST   
  REST -> SORT_BLOCK, REST 
  REST -> IF_BLOCK, REST   
  REST -> LAST_BLOCK

  LAST_BLOCK -> SELECT_BLOCK
  LAST_BLOCK -> BY_BLOCK
  LAST_BLOCK -> SORT_BLOCK
  LAST_BLOCK -> IF_BLOCK

  
  FROM_BLOCK -> FROM_EXPR

  FROM_EXPR -> FROM_SELECTOR FROM_EXPR
  FROM_EXPR -> FROM_SELECTOR

  // Wildcard and brace expansion
  FROM_SELECTOR ~~> FROM ... FROM   


  // Standard input
  FROM -> ASSIGN_NAME=- OPTIONS
  FROM -> - OPTIONS

  FROM -> ASSIGN_NAME=FILE_PATH OPTIONS
  FROM -> FILE_PATH OPTIONS

  
  OPTIONS -> -dCHAR OPTIONS
  OPTIONS -> --delimiter=CHAR OPTIONS

  OPTIONS -> -sCHAR OPTIONS
  OPTIONS -> --secondary-delimiter=CHAR OPTIONS

  OPTIONS -> -n OPTIONS  
  OPTIONS -> --named OPTIONS

  OPTIONS -> -N OPTIONS   
  OPTIONS -> --not-named OPTIONS

  OPTIONS ->
  
  
  SELECT_BLOCK -> SELECT_EXPR
  BY_BLOCK -> by SELECT_EXPR
  SORT_BLOCK -> sort SELECT_EXPR                                                                                                                              
  IF_BLOCK -> if ARITHMETIC_EXPR
  
                                                                                                                 
  ARITHMETIC_EXPR -> ATOM  

  ARITHMETIC_EXPR -> ARITHMETIC_EXPR OPERATOR ARITHMETIC_EXPR
  ARITHMETIC_EXPR -> (ARITHMETIC_EXPR)

  // Logical negation
  ARITHMETIC_EXPR -> ! ARITHMETIC_EXPR
  // Number negation
  ARITHMETIC_EXPR -> - ARITHMETIC_EXPR
  

  SELECT_EXPR -> ATOM_SELECTOR SELECT_EXPR                                                                                                                                     
  SELECT_EXPR -> ATOM_SELECTOR
  
  // Wildcard and brace expansion
  ATOM_SELECTOR ~~> ATOM ... ATOM 

  
  ATOM -> pi   
  ATOM -> e
  ATOM -> true
  ATOM -> false

  // e.g. 1.0, "text", 'text', 1
  ATOM -> CONSTANT
  // e.g. &1.1
  ATOM -> SYMBOL_NAME

  ATOM -> $(ARITHMETIC_EXPR)
  ATOM -> AGGREGATE_FUNCTION(SELECT_EXPR)
  ATOM -> ONEARG_FUNCTION(ARITHMETIC_EXPR)


  // # is not a char:
  // Two atoms can be written without whitespace
  // and their values will be String appended 
  // if the right atom begins with a char, 
  // which can't be a part of the left atom.
  //
  // E.g. if the left atom is a number constant, 
  // and the right atom is a String constant 
  // beginning with a quote char,
  // the left atom value will be converted to the String
  // and prepended to the right atom value.
  //
  // This rule doesn't apply inside ARITHMETIC_EXPR
  ATOM ~~> ATOM#ATOM     

  
  // Converts all values to the String type and appends them.
  AGGREGATE_FUNCTION -> cat

  // Returns the number of values.
  AGGREGATE_FUNCTION -> count

  AGGREGATE_FUNCTION -> min
  AGGREGATE_FUNCTION -> max
  AGGREGATE_FUNCTION -> sum
  AGGREGATE_FUNCTION -> avg

  
  // All trigonometric functions are in radians.
  ONEARG_FUNCTION -> sin
  ONEARG_FUNCTION -> cos
  ONEARG_FUNCTION -> tan
  
  ONEARG_FUNCTION -> asin
  ONEARG_FUNCTION -> acos
  ONEARG_FUNCTION -> atan
  
  ONEARG_FUNCTION -> sinh
  ONEARG_FUNCTION -> cosh
  ONEARG_FUNCTION -> tanh
  
  ONEARG_FUNCTION -> asinh
  ONEARG_FUNCTION -> acosh
  ONEARG_FUNCTION -> atanh
  
  ONEARG_FUNCTION -> exp
  ONEARG_FUNCTION -> sqrt
  
  // Converts a value to the String type and returns its length.
  ONEARG_FUNCTION -> size

  ONEARG_FUNCTION -> to_string
  
  ONEARG_FUNCTION -> negate
  ONEARG_FUNCTION -> abs
  ONEARG_FUNCTION -> signum
  
  ONEARG_FUNCTION -> truncate
  ONEARG_FUNCTION -> ceiling
  ONEARG_FUNCTION -> floor
  
  ONEARG_FUNCTION -> even
  ONEARG_FUNCTION -> odd

  
  // A in B means A is a substring of B.
  OPERATOR -> in
  
  OPERATOR -> *
  OPERATOR -> /

  // General power
  OPERATOR -> **    
  // Natural power
  OPERATOR -> ^     
  
  // Integer division truncated towards minus infinity
  // (x div y)*y + (x mod y) == x
  OPERATOR -> div
  OPERATOR -> mod
  
  // Integer division truncated towards 0
  // (x quot y)*y + (x rem y) == x  
  OPERATOR -> quot
  OPERATOR -> rem

  // Greatest common divisor
  OPERATOR -> gcd
  // Least common multiple
  OPERATOR -> lcm
  
  // String append
  OPERATOR -> ++    
  
  OPERATOR -> +
  OPERATOR -> -
  
  OPERATOR -> <=
  OPERATOR -> >=
  OPERATOR -> <
  OPERATOR -> >
  OPERATOR -> !=
  OPERATOR -> ==
  
  OPERATOR -> ||
  OPERATOR -> &&
\end{verbatim}

Each command is made from blocks separated by a comma. There are these types of blocks.
\begin{itemize}
    \item From block
    \item Select block
    \item If block
    \item By block
    \item Sort block
\end{itemize}

The first block is always the from block. If the block after the first block is without a specifier (\icode{if}, \icode{by}, or \icode{sort}), then it is the select block. Otherwise, it is a block specified by the specifier.

The from block accepts a specific grammar (as specified in the grammar description), the select, the by, and the sort block accept the select expression (\icode{SELECT\_EXPR} in the grammar), 
and the if block accepts the arithmetic expression (\icode{ARITHMETIC\_\allowbreak EXPR} in the grammar).

Every source data file has a reference number based on its position in the from block and may have multiple names---the assign name, the name given to the source data file by \icode{ASSIGN\_NAME=FILE\_PATH} syntax in the from block, and  
the default name, which is given by the path to the file or \icode{-} in the case of the standard input in the from block.

Each column of a source data file has a reference number based on its position in it and may have a name (if the named option is enabled for the given source file).

If a source data file with the reference number \icode{M} (numbering input files from 1) has a name \icode{XXX}, 
its columns can be addressed by \icode{\&M.N} or \icode{XXX.N}, where \icode{N} is the reference number of a column (numbering columns from 1).  
If the named option is enabled for the input file and a column has the name \icode{NAME}, it can also be addressed by \icode{\&M.NAME} or \icode{XXX.NAME}.

We call the address a symbol name---\icode{SYMBOL\_NAME} in the grammar description.

If there is a collision in naming (some symbol name addresses more than one column), then the behavior is undefined.

\subsubsection{Exotic chars}
Some chars cannot be in unquoted symbol names---exotic chars. For simplicity, we can suppose, they are all non-alphanumerical chars excluding \icode{-}, \icode{.}, \icode{\&}, and \icode{\_}.  
Also the first char of a symbol name must be non-numerical and must not be \icode{-} or \icode{.}  to not be considered as an exotic char.

It is possible to use a symbol name with exotic chars using \icode{`} quote---like \icode{`EXOTIC SYMBOL NAME`}. 

\subsubsection{Quote chars}
There are 3 quote chars (\icode{`}, \icode{"} and \icode{'}) used in LSQL. \icode{"} and \icode{'} are always quoting a \icode{String}. The \icode{`} quote char is used for quoting symbol names.

These chars can be used for \icode{String} appending. If two atoms inside \icode{SELECT\_\-EXPR} are written consecutively without whitespace and the left atom ends by a quote char or the right begins by a quote char, 
they will be converted to the \icode{String} and will be \icode{String} appended. 
For example, \icode{\&1.1"abc"} means: convert the value of \icode{\&1.1} to the \icode{String} and append it to the \icode{String} constant \icode{abc}.

\subsubsection{Constants}
Constants are in the grammar description as \icode{CONSTANT}. In the following section, we speak only about these constants and not about built-in constant values like \icode{pi} or \icode{true}.

There are 3 datatypes of constants. \icode{String}, \icode{Double}, and \icode{Int}. 
Every string quoted in \icode{"} chars or \icode{'} chars in an LSQL command is always tokenized as a \icode{String} constant. 
Numbers fully matching the POSIX ERE \icode{[0-9]+} are considered \icode{Int} constants and numbers fully matching the POSIX ERE \icode{[0-9]+\textbackslash{}.[0-9]+} \icode{Double} constants.

\subsubsection{Operator associativity and precedence}
All operators are right-to-left associative.

The following list outlines the precedence of the \icode{lsql-csv} infix operators. The lower the precedence number, the higher the priority.
\begin{center}
\begin{tabular}{|c|c|}
\hline
Precedence number & Operator\\
\hline\hline
1 & \icode{in}, \icode{**}, \icode{\^}\\
\hline
2 & \icode{*}, \icode{/}, \icode{div}, \icode{quot}, \icode{rem}, \icode{mod}, \icode{gcd}, \icode{lcm}\\
\hline
3 & \icode{++}, \icode{+}, \icode{-}\\
\hline
4 & \icode{\textless=}, \icode{\textgreater=}, \icode{\textless}, \icode{\textgreater}, \icode{!=}, \icode{==}\\
\hline
5 & \icode{$\vert\vert$}, \icode{\&\&}\\
\hline
\end{tabular}
\end{center}


\subsubsection{Select expression}
Select expressions are in the grammar description as \icode{SELECT\_EXPR}.
They are similar to the \icode{bash} expressions \cite{bash-reference-manual}. They are made by atom selector expressions (\icode{ATOM\_SELECTOR}) separated by whitespaces. 
These expressions are wildcard and brace expanded to atoms (\icode{ATOM}) and are further processed as they were separated by whitespace.

Wildcards and brace expansion expressions are only evaluated and expanded in unquoted parts of the atom selector expression, which aren't part of an inner arithmetic expression.

Every atom selector expression can consist:
\begin{itemize}
    \item A wildcard (Each wildcard is expanded against the symbol name list.
	    If no symbol name matching the wildcard is found, the wildcard is expanded to itself.);
    \item A \icode{bash} brace expansion expression (e.g. \icode{\{22..25\}} $\rightarrow$ \icode{22 23 24 25}) \cite{bash-reference-manual};
    \item An arithmetic expression in \icode{\$(expr)} format;
    \item A call of an aggregate function \icode{AGGREGATE\_FUNCTION(SELECT\_EXPR)}---\allowbreak{}the\-re cannot be any space after \icode{FUNCTION};
    \item A call of a one-argument function \icode{ONEARG\_FUNCTION(ARITHMETIC\_EXPR)}---there cannot be any space after \icode{FUNCTION};
    \item A constant;
    \item A built-in constant value;
    \item A symbol name.
\end{itemize}

Please, keep in mind, that operators (\icode{OPERATOR}) must be put inside arithmetic expressions.

\subsubsection{Arithmetic expression}
Arithmetic expressions are in the grammar description as \icode{ARITHMETIC\_EXPR}.

The expressions use mainly the classical \icode{awk} style of expressions \cite{awk-reference-manual}.  
You can use here operators (\icode{OPERATOR}) keywords \icode{\textgreater}, \icode{\textless}, \icode{\textless=}, \icode{\textgreater=}, \icode{==}, \icode{$\vert\vert$}, \icode{\&\&}, \icode{+}, \icode{-}, \icode{*}, \icode{/}\ldots{}

\subsubsection{Select blocks}
Select blocks are referred in the grammar description as \icode{SELECT\_BLOCK}.
These blocks determine the output. They accept the select expression.

There must be at least one select block in an LSQL command, which refers to at least one symbol name, or the behavior is undefined.

Examples of select blocks:
\begin{verbatim}
&1.[3-6]
\end{verbatim}
This will print the 3rd, 4th, 5th, and 6th columns from the first file if the first file has at least 6 columns.

\begin{verbatim}
ax*.{6..4}
\end{verbatim}
This will print the 6th, the 5th, and the 4th columns from all files whose name begins with ax if the files have at least 6 columns.

\subsubsection{From blocks}
These blocks are in the grammar description as \icode{FROM\_BLOCK}.
There must be exactly one from block at the beginning of an LSQL command. 

The from block contains input file paths (or \icode{-} in the case of the standard input), and optionally their assign name \icode{ASSIGN\_NAME}. 

You can use the wildcards and the curly bracket expansion as you were in the \icode{bash} to refer input files \cite{bash-reference-manual}. 
If there is a wildcard with an assign name \icode{NAME} matching more than one input file, the input files will be given assign names \icode{NAME}, \icode{NAME1}, \icode{NAME2}\ldots{}
If there is a wildcard, that matches to no file, it is expanded to itself.

If \icode{FILE\_PATH} is put inside \icode{`} quotes, no wildcard or expansion logic applies to it.

You can also add custom attributes to input files in the format \icode{FILE\_PATH -aX --attribute=X -b}. The attributes will be applied to all files which will be matched against \icode{FILE\_PATH}.
The custom attributes are referred to as \icode{OPTIONS} in the grammar description.

Examples:
\begin{verbatim}
/etc/{passwd,group}
\end{verbatim}

This will select \icode{/etc/passwd} and \icode{/etc/group} files. They can be addressed either as \icode{\&1} or \icode{/etc/passwd}, and \icode{\&2} or \icode{/etc/group}.

\begin{verbatim}
passwd=/etc/passwd
\end{verbatim}
This will select \icode{/etc/passwd} and set its assign name to \icode{passwd}. It can be addressed as \icode{\&1}, \icode{passwd}, or \icode{/etc/passwd}.


\paragraph{Possible custom attributes}

\begin{verbatim}
-n
--named
\end{verbatim}
Enables the first-line naming convention for an input CSV file---first-line with column names.

\begin{verbatim}
-N
--not-named
\end{verbatim}
You can also set the exact opposite to an input file. This can be useful if you change the default behavior.

\begin{verbatim}
-dCHAR
--delimiter=CHAR
\end{verbatim}
This changes the primary delimiter of an input file.

\begin{verbatim}
-sCHAR
--secondary-delimiter=CHAR
\end{verbatim}
This changes the secondary delimiter of an input file.

Example:
\begin{verbatim}
/etc/passwd -d:
\end{verbatim}
This will select \icode{/etc/passwd} and set its delimiter to \icode{:}.

Currently, commas and \icode{CHAR}s, which are also quotes in LSQL, are not supported as a delimiter or a secondary delimiter in \icode{FILE\_PATH} custom attributes.

\subsubsection{If blocks}
These blocks are in the grammar description as \icode{IF\_BLOCK}.
They always begin with the \icode{if} keyword. 
They accept arithmetic expressions, which should be convertible to \icode{Bool}: 
either \icode{String} \icode{false}/\icode{true}, \icode{Int} (\icode{0} \icode{false}, anything else \icode{true}), or \icode{Bool}. 

Rows with the arithmetic expression converted to \icode{Bool} \icode{true} are printed or aggregated, and 
rows with the arithmetic expression converted to \icode{Bool} \icode{false} are skipped.

Filtering is done before the aggregation.

You can imagine the if block as the \icode{WHERE} clause in SQL.

\subsubsection{By blocks}
By blocks are referred in the grammar description as \icode{BY\_BLOCK}.
These blocks always begin with the \icode{by} keyword. They accept the select expression. 

There can be only one by block in a whole LSQL command.

The by block is used to group the resulting set by the given atoms for the evaluation by an aggregate function.
The by block is similar to the \icode{GROUP BY} clause in SQL. 

There must be at least one aggregate function in the select block if the by block is present. Otherwise, the behavior is undefined.

If there is an aggregate function present without the by block present in an LSQL command, the aggregate function runs over all rows at once.

\subsubsection{Sort blocks}
These blocks are in the grammar description as \icode{SORT\_BLOCK}.
It begins with the \icode{sort} keyword. They accept the select expression.

The sort block determines the order of the final output---given atoms are sorted in ascending order.
If there is more than one atom in the sort block (\icode{A}, \icode{B}, \icode{C}\ldots), the data is first sorted by \icode{A} 
and in the case of ties, the atoms (\icode{B}, \icode{C}\ldots) are used to further refine the order of the final output.

You can imagine the sort block as the \icode{ORDER BY} clause in SQL.

There can be only one sort block in the whole command.

