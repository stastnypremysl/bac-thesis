\chapter*{Introduction}
\addcontentsline{toc}{chapter}{Introduction}

Database refers to a set of related data accessed through the use of a database management system \cite{enwiki-database}.
CSV files (Comma Separated Value files) are a common way of exchanging and converting data between various spreadsheet programs \cite{rfc4180}.
Through this definition, we can see even a simple collection of CSV files accessed through some programs may be seen as a database itself.


SQL (Structured Query Language) is a language used to manage data, especially in a relational database management system \cite{enwiki-sql}.
It was first introduced in the 1970s \cite{enwiki-sql} and is one of the most used query languages. 
Despite being standardized in 1986 by the American National Standards Institute \cite{ANSI-X3.135-1986} and
in 1987 by the International Organization for Standardization \cite{ISO9075-1987}, 
there are virtually no implementations that adhere to it fully \cite{enwiki-sql}.
Standard SQL is a typed language (every data value belongs to some data type) \cite{ISO9075-2023} and the language
design is therefore not very suitable for type-less databases like a collection of CSV files.
Despite that, there are some implementations of SQL (e.g. \icode{q} \cite{q}, \icode{CSV SQL} \cite{csv-sql}, \icode{trdsql} \cite{trdsql}, or \icode{csvq} \cite{csvq}), which tries to implement SQL on CSV files.

SQL itself requires a large amount of text to be written for running even simple queries and the Unix ecosystem misses a tool\footnote{Or author does not know about it.}, 
that would allow running short enough queries over CSV files with similar semantics to SQL. And this is the reason, why \icode{lsql-csv} was created.

\icode{lsql-csv} is a tool for small CSV file data querying from a shell with short queries. It makes it possible to work with small CSV files like with a read-only relational databases.
The tool implements a new language LSQL similar to SQL, specifically designed for working with CSV files in a shell.

Haskell is a language with great features for working with the text \cite{practical-haskell} and therefore it was selected for the task of implementation of \icode{lsql-csv}.
