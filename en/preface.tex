\chapter*{Introduction}
\addcontentsline{toc}{chapter}{Introduction}

Database refers to a set of related data accessed through the use of a database management system \cite{enwiki-database}.
CSV files (Comma Separated Value files) are common way of exchanging and converting data between various spreadsheet programs \cite{rfc4180}.
Through this definition, we can see even a simple collection of CSV files accessed through some programs may be seen as an database itself.


SQL (Structured Query Language) is a language used to manage data, especially in a relational database management system \cite{enwiki-sql}.
It was first introduced at 1970s \cite{enwiki-sql} and is one of the most used query languages. 
Despite being standardized at 1987 by International Organization for Standardization \cite{ISO9075-1987}, 
there are virtually no implementations adhere to it fully \cite{enwiki-sql}.
Standard SQL is a typed language (Every data value belongs to some data type) \cite{ISO9075-2023} and the language
design is therefore not much suitable for type-less databases like a collection of CSV files.
Despite that, there are some implementations of SQL (eg. q \cite{q}), which tries to implement SQL on CSV files.

SQL itself requires large amount of text to be written for running even a simple queries and Unix ecosystem misses a tool\footnote{Or author don't know about it}, 
which would allowed running a short enough queries over CSV files with similar semantics like SQL has. And this is the reason, why lsql-csv was created.

Haskell is a language with great features for working with text \cite{practical-haskell} and therefore it was selected for the task of implementation of lsql-csv.

lsql-csv is a tool for CSV files data querying from shell with short queries. It makes possible to work with small CSV files like with read-only relational database.
The tool implements a new language LSQL similar to SQL, which is type-less, specifically designed for working with CSV files in shell.
