
\chapter{Alternative Solutions}
What other solutions are there for the given problem? What other approaches can we use, when we are dealing with queries over CSV files? 
The following chapter is about alternative approaches to the problem.

\section{Using SQL database}
The first obvious solution is importing the dataset to some standard SQL database and doing the queries over it. 
This approach requires the definition of the schema into which the data will be imported. This is an overhead, which isn't
always advantageous to pay as we might need only a simple query to be run over it.
The data import might take also a much longer time than a simple \icode{lsql-csv} invocation.

But it might be a very favorable solution, if we need large dataset joins, need a complex query execution or a large amount of simple queries run over it.

By using a standard SQL database you gain the advantage of better performance, typed datasets, indexes, and a larger number of built-in functions.

\section{Using standard Unix tools}
It is possible to do a large amount of work just by using \icode{awk}, \icode{join}, \icode{sort}\ldots{}
For example:

\begin{verbatim}
lsql-csv -d: '-, &1.*, if &1.3 >= 1000' </etc/passwd
\end{verbatim}

This query might be rewritten to \icode{awk}:
\begin{verbatim}
awk -F: '{ if($3 >= 1000){ print $0 }}' </etc/passwd
\end{verbatim}

The main advantage of \icode{lsql-csv} is that it handles some more complex queries more easily. For example:
\begin{verbatim}
lsql-csv -d: '/etc/{passwd,group}, &1.1 &2.1, if &1.4 == &2.3'
\end{verbatim}

This is a simple join query. When written using standard Unix tools, it is:
\begin{verbatim}
sort -t: -k3,3 /etc/group >/tmp/group.sort
sort -t: -k4,4 /etc/passwd >/tmp/passwd.sort
join -t: -14 -23 /tmp/passwd.sort /tmp/group.sort | cut -d: -f2,8 
\end{verbatim}

As demonstrated, the \icode{lsql-csv} variant is 
more readable and shorter\footnote{It is possible, that it can be written in shorter and more readable form, but probably not more than the \icode{lsql-csv} variant.}.

It should be also noted, that \icode{lsql-csv} join has $\mathcal{O}(nm)$ time complexity, while standard Unix tools have for join written above
$\mathcal{O}(n\log n + m\log m)$ time complexity, so for larger datasets, it might be more beneficial to use them.

\section{By using SQL implementation for CSV files}
There are many projects implementing SQL on CSV files. For example:
\begin{itemize}
    \item \icode{q} \cite{q}
    \item \icode{CSV SQL} \cite{csv-sql} 
    \item \icode{trdsql} \cite{trdsql}
    \item \icode{csvq} \cite{csvq}
\end{itemize}

It is possible to use them to do the job directly with SQL. 

The advantage of it is you do not have to learn a new language if you already know standard SQL. 
The disadvantage is that the queries will be probably longer than would be with \icode{lsql-csv}.

\section{By using general-purpose programming language}
It is not hard to parse and process CSV files with a general-purpose programming language (for example Python).

The advantage of this solution is a much greater flexibility of what you can do with the CSV files.
The large disadvantage is, that it will take too much code to be written for any query.

