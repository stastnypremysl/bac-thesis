\chapter{Developer documentation}
This chapter is for potential developers of the project.

\section{Project building and testing}
The project have two ways of building. One way is through Makefile and second through cabal. 
By running the following command it is generated build folder and lsql-csv binary under it.
\begin{verbatim}
make
\end{verbatim}
 
It is necessary to have all Haskell dependencies (\textit{Parsec} ($\geq 3.1$ $<3.2$), \textit{Glob} ($\geq 0.10$ $<0.11$), 
\textit{base} ($\geq 4.9$ $<4.17$), \textit{text} ($\geq 1.2$ $<1.3$) and \textit{containers} ($\geq 0.5$ $<0.7$)) installed.
The package boundaries given are identical to cabal bounderies. Also it is necessary, you had \textit{GHC} ($\geq 8 <9.29$) installed.

Second way of building is through cabal, which handles all dependencies for you.

\begin{verbatim}
cabal build
\end{verbatim}


The project unit tests require building through \textit{Makefile} and is called by:
\begin{verbatim}
make test
\end{verbatim}
It should always succeed before any commit is made.

It is possible to generate Haddock developer documentation by calling:

\begin{verbatim}
cabal haddock
\end{verbatim}

The documentation contains comments on all exported functions. The documentation can be alternatively generated
by running:

\begin{verbatim}
make docs
\end{verbatim}

It generates html documentation under build folder.

\section{\textit{String} vs \textit{Data.Text}}
As there is a long term discussion in Haskell community whether \textit{String} or \textit{Data.Text} should be used as the 
primary representation for text, I would like to emphasize that in this project, String is used as a primary
representation for text.

\section{Project layout}

The project is split to:
\begin{enumerate}
  \item a library, which contains almost all the logic and is placed under
       \textit{src} folder of the project

  \item the \textit{main}, which contains one source file
      with \textit{Main}, which is the entry point for \textit{lsql-csv} binary. 
      It parses the arguments, checks, whether help optional argument was called or no argument at all was given, 
      and either shows help or further call run from \textit{Lsql.Csv.Main} from library.
\end{enumerate}

Library is split into 5 different namespaces. Its usage is not strictly defined, but this can be said
\begin{itemize}
    \item \textit{Lsql.Csv} -- This namespace contains the entry point
    \item \textit{Lsql.Csv.Core} -- This namespace contains the logic of evaluation
    \item \textit{Lsql.Csv.Lang} -- This namespace contains parsers
    \item \textit{Lsql.Csv.Lang.From} -- This namespace cointains parsers for from block
    \item \textit{Lsql.Csv.Utils} -- This namespace contains helper functions
\end{itemize}

\section{Modules}
The following section is a summary of all modules of the library.
\begin{itemize}
    \item \textit{Lsql.Csv.Core.BlockOps} -- This module contains the \textit{Block} definition and functions for getting specific types of blocks from list of \textit{Block}.
    \item \textit{Lsql.Csv.Core.Evaluator} -- This module contains the evaluator of \textit{lsql-csv} program.
    \item \textit{Lsql.Csv.Core.Functions} -- This module contains the syntactic tree definition and helper functions for its evaluation.
    \item \textit{Lsql.Csv.Core.Symbols} -- This module contains the definition of \textit{Symbol}, \textit{SymbolMap} and helper functions.
    \item \textit{Lsql.Csv.Core.Tables} -- This module contains the definition of \textit{lsql-csv} 
        data types, classes over them and types \textit{Table} and \textit{Column} and functions over them for manipulation of them.
    \item \textit{Lsql.Csv.Lang.Args} -- A module for command line argument parsing.
    \item \textit{Lsql.Csv.Lang.BlockChain} -- This module contains a main parser of blocks other than from block.
    \item \textit{Lsql.Csv.Lang.BlockSeparator} -- This module contain a preprocessor parser, which splits command into list of strings - one string per one block.
    \item \textit{Lsql.Csv.Lang.Options} -- This module implements a common \textit{Option} type and it is parsers for from blocks and command line optional arguments.
    \item \textit{Lsql.Csv.Lang.Selector} -- This module implements the selector expression and arithmetic expression parsers.
    \item \textit{Lsql.Csv.Lang.From.Block} -- This module contains the from block parser. It loads initial \textit{SymbolMap}.
    \item \textit{Lsql.Csv.Lang.From.CsvParser} -- This module contains \textit{CsvParser} called by \textit{parseFile}, which loads input CSV files.
    \item \textit{Lsql.Csv.Main} -- This module contains the starting point for \textit{lsql-csv} evaluation.
    \item \textit{Lsql.Csv.Utils.BracketExpansion} -- This module contains the curly bracket (braces) expansion implementation.
    \item \textit{Lsql.Csv.Utils.CsvGenerator} -- This module contains the CSV generator for the output.
\end{itemize}

\section{Entry point}
The entry point is in module \textit{Lsql.Csv.Main} in function \textit{run}. 
The function first calls preprocesor in \textit{Lsql.Csv.Lang.BlockSeparator}, which splits command into list of strings. Then \textit{SymbolMap} with input is loaded using \textit{Lsql.Csv.Lang.From.Block} and after that
rest of blocks are parsed using \textit{Lsql.Csv.Lang.BlockChain}. Program is then evaluated using
\textit{Lsql.Csv.Core.Evaluator} and finally output generated by \textit{Lsql.Csv.Utils.CsvGenerator}.



