\chapter{Developer documentation}
This chapter is for potential developers of the project.

\section{Project building and testing}
The project has two ways of building. The first way is through \icode{Makefile} and the second is through Cabal. 
By running the following command it generates the \icode{build} folder and the \icode{lsql-csv} binary under it.
\begin{verbatim}
make
\end{verbatim}
 
It is necessary to have all Haskell dependencies (\icode{Parsec} ($\geq 3.1$ $<3.2$), \icode{Glob} ($\geq 0.10$ $<0.11$), 
\icode{base} ($\geq 4.9$ $<4.20$), \icode{text} ($\geq 1.2$ $<2.2$) and \icode{containers} ($\geq 0.5$ $<0.8$)) installed.
The package boundaries given are identical to the Cabal boundaries. Also, it is necessary, that you have \icode{GHC} ($\geq 8 <9.29$) installed.

The second way of building is through Cabal, which handles all dependencies for you.

\begin{verbatim}
cabal build
\end{verbatim}


The project unit tests require building through \icode{Makefile} and are called by:
\begin{verbatim}
make test
\end{verbatim}
It should always succeed before any commit to the project repository is made.

It is possible to generate Haddock developer documentation by calling:
\begin{verbatim}
cabal haddock
\end{verbatim}

The documentation contains comments on all exported functions. The documentation can be alternatively generated
by running:

\begin{verbatim}
make docs
\end{verbatim}
It generates HTML documentation under the \icode{build} folder.

The package is also published at \url{https://hackage.haskell.org/package/lsql-csv} in the Hackage public repository. 
The generated documentation is fully browseable there.

\section{\icode{String} vs \icode{Data.Text}}
As there is a long-term discussion in the Haskell community about whether \icode{String} or \icode{Data.Text} should be used as the 
primary representation of text, I would like to emphasize that in this project, \icode{String} is used as the primary
representation of text.

\section{Project layout}

The project is split into:
\begin{enumerate}
  \item a library, which contains almost all the logic and is placed under
       \icode{src} folder of the project

  \item the \icode{main}, which contains one source file
      with \icode{Main}, which is the entry point for \icode{lsql-csv} binary. 
      It parses the arguments, and checks, whether the help optional argument was called and whether no argument at all was given, 
     and either displays the usage message or further call \icode{run} from \icode{Lsql.Csv.Main} in the library---the evaluation entry point.
\end{enumerate}

The library is split into 5 namespaces. Their usage is not strictly defined, but this can be said:
\begin{itemize}
    \item \icode{Lsql.Csv} -- This namespace contains the starting point for an \icode{lsql-csv} evaluation.
    \item \icode{Lsql.Csv.Core} -- This namespace contains the logic of the evaluation.
    \item \icode{Lsql.Csv.Lang} -- This namespace contains parsers for the blocks other than the from block.
    \item \icode{Lsql.Csv.Lang.From} -- This namespace contains parsers for the from block and for CSV files.
    \item \icode{Lsql.Csv.Utils} -- This namespace contains helper functions.
\end{itemize}


\section{Modules}
The following section is a summary of all modules of the library.
\begin{itemize}
    \item \icode{Lsql.Csv.Core.BlockOps} -- This module contains the \icode{Block} definition representing an LSQL command block and functions for getting a specific type of block from a list of \icode{Block}.
    \item \icode{Lsql.Csv.Core.Evaluator} -- This module contains the evaluator of an \break\icode{lsql-csv} command.
    \item \icode{Lsql.Csv.Core.Functions} -- This module contains the syntactic tree definition and helper functions for its evaluation.
    \item \icode{Lsql.Csv.Core.Symbols} -- This module contains the definition of \icode{Symbol}, \icode{SymbolMap}, and helper functions for working with them. \icode{SymbolMap} is one of the representations of input data.
    \item \icode{Lsql.Csv.Core.Tables} -- This module contains the definition of \icode{Value}, \icode{Table}, and \icode{Column}, class instancies over them, functions for manipulation of them, and \icode{Boolable} class definition.
    \item \icode{Lsql.Csv.Lang.Args} -- A module for command-line argument parsing.
    \item \icode{Lsql.Csv.Lang.BlockChain} -- This module contains the main parser of the blo\-cks other than the from block.
    \item \icode{Lsql.Csv.Lang.BlockSeparator} -- This module contains the preprocessor parser, which splits an LSQL command into a list of \icode{String}---one \icode{String} per block.
    \item \icode{Lsql.Csv.Lang.Options} -- This module implements the common \icode{Option} type for the from block custom attributes representation and for the co\-mmand-line optional arguments representation, and its parsers.
    \item \icode{Lsql.Csv.Lang.Selector} -- This module implements the selector expression parser and the arithmetic expression parser.
    \item \icode{Lsql.Csv.Lang.From.Block} -- This module contains the from block parser. It loads the initial \icode{SymbolMap} with input data.
    \item \icode{Lsql.Csv.Lang.From.CsvParser} -- This module contains the \icode{CsvParser} called by the \icode{parseFile}, which loads input CSV files.
    \item \icode{Lsql.Csv.Main} -- This module contains the starting point for an \icode{lsql-csv} evaluation.
    \item \icode{Lsql.Csv.Utils.BracketExpansion} -- This module contains the curly bra\-cket (braces) expansion implementation.
    \item \icode{Lsql.Csv.Utils.CsvGenerator} -- This module contains the CSV generator for the output.
\end{itemize}


\section{Evaluation entry point}
The evaluation entry point is in the module \icode{Lsql.Csv.Main} in the function \icode{run}. 
The function first calls the preprocessor in \icode{Lsql.Csv.Lang.BlockSeparator}, which splits the input command into a list of \icode{String}---one \icode{String} per block. 
Then the \icode{SymbolMap} with input data is loaded using \icode{Lsql.\allowbreak Csv.\allowbreak Lang.\allowbreak From.\allowbreak Block} and after that,
the rest of the blocks are parsed using \icode{Lsql.\allowbreak Csv.\allowbreak Lang.\allowbreak BlockChain}. 

The loaded data are then processed using the evaluator in
\icode{Lsql.\allowbreak Csv.\allowbreak Core.\allowbreak Evaluator} according to the parsed command and finally, the output is generated by \icode{Lsql.\allowbreak Csv.Utils.\allowbreak CsvGenerator}.


\section{Unit tests}
There was created many unit tests for testing the functionality of the \icode{lsql-csv} binary.

All unit tests are shell scripts. Each test invokes the \icode{lsql-csv} binary in the \icode{build} project folder
with predefined input data, predefined optional arguments, and a predefined LSQL command, and compares
the output of the \icode{lsql-csv} invocation with a predefined expected output.
If the expected output is the same as the actual output, the test exits with code \icode{0}. 
If the expected output is different from the actual output, the test exits with code \icode{1}. 

All unit tests are split into subfolders of the project folder \icode{tests}.
There are these subfolders:
\begin{enumerate}
    \item \icode{basics} -- A basic functionality tests.
    \item \icode{blocks} -- Tests of all block types and their combinations.
    \item \icode{examples} -- Tests of a majority of examples in the user documentation.
    \item \icode{options} -- Tests of the command-line optional arguments, and the from block custom attributes.
    \item \icode{operators} -- Tests of operators, their precedence, and associativity.
    \item \icode{onearg-functions} -- Tests of one-argument functions.
    \item \icode{aggregate-functions} -- Tests of all aggregate functions.
\end{enumerate}

For each test subfolder \icode{FOLDER}, there is a \icode{Makefile} target \icode{test-FOLDER}, which indirectly invokes all the tests in the subfolder. 
The \icode{Makefile} target \icode{test} indirectly invokes the tests in all subfolders.
