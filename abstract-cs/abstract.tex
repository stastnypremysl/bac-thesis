\documentclass[a4paper]{article}

\let\pdfcreationdate=\creationdate
\usepackage[a-2u]{pdfx}

\usepackage[utf8]{inputenc}
\usepackage{lmodern}

\hypersetup{
	unicode,
	pdfauthor={Přemysl Šťastný},
	pdftitle={Nástroj lsql-csv na zpracování CSV souborů z příkazového řádku - abstrakt},
	pdfsubject={Nástroj lsql-csv na zpracování CSV souborů z příkazového řádku - abstrakt},
	pdfkeywords={abstract, relační databáze, CSV, SQL, Haskell, dotazovací jazyk, unixová filozofie},
	pdfproducer={LaTeX},
	pdfcreator={xelatex}
}

\begin{document}
lsql-csv je nástroj pro provádění krátkých dotazů nad daty malých CSV souborů z shellu. 
Díky němu je možné pracovat s malými CSV soubory jako s read-only relační databází.
Nástroj implementuje nový jazyk LSQL podobný SQL, speciálně navržený pro práci se CSV soubory v shellu.
Designovým cílem LSQL je být lapidárnějším jazykem než SQL.
Účelem jeho návrhu je umožnit uživateli rychle psát jednoduché dotazy přímo do terminálu.
\end{document}
