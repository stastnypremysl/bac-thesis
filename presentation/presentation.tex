\documentclass{beamer}
%
% Choose how your presentation looks.
%
% For more themes, color themes and font themes, see:
% http://deic.uab.es/~iblanes/beamer_gallery/index_by_theme.html
%
\mode<presentation>
{
  \usetheme{default}      % or try Darmstadt, Madrid, Warsaw, ...
  \usecolortheme{default} % or try albatross, beaver, crane, ...
  \usefonttheme{default}  % or try serif, structurebold, ...
  \setbeamertemplate{navigation symbols}{}
  \setbeamertemplate{caption}[numbered]
} 

\usepackage[czech]{babel}
\usepackage[utf8]{inputenc}
\usepackage[T1]{fontenc}
\usepackage{fontspec}

\def\icode#1{\texttt{#1}}


\title[Nástroj \icode{lsql-csv} na zpracování CSV souborů z příkazového řádku]{Nástroj \icode{lsql-csv} na zpracování CSV souborů z příkazového řádku}
\author{Přemysl Šťastný}
\institute{Univerzita Karlova}
\date{28.06.2024}

\begin{document}

\begin{frame}
  \titlepage
\end{frame}

% Uncomment these lines for an automatically generated outline.
%\begin{frame}{Outline}
%  \tableofcontents
%\end{frame}

\section{Úvod}

\begin{frame}{Úvod}

\begin{itemize}
  \item \icode{lsql-csv} je nástroj pro provádění krátkých dotazů nad daty malých CSV souborů z shellu
  \item Implementuje nový jazyk LSQL podobný SQL, speciálně navržený pro práci se CSV soubory v shellu
  \item Designovým cílem LSQL je být lapidárnějším jazykem než SQL. 
  \item Účelem jeho návrhu je umožnit uživateli rychle psát jednoduché dotazy přímo do terminálu.
\end{itemize}


\end{frame}


\section{Příklad: Hello World}

\begin{frame}[fragile]{Příklad: Hello World}

\begin{verbatim}
lsql-csv '-, &1.2 &1.1' <<- EOF
World,Hello
EOF
\end{verbatim}

\begin{verbatim}
Hello,World
\end{verbatim}

\end{frame}

\section{Příklad: Jednoduché filtrování}
\begin{frame}[fragile]{Příklad: Jednoduché filtrování}

\begin{verbatim}
lsql-csv -d: '-, &1.*, if &1.3 >= 1000' </etc/passwd
\end{verbatim}

\begin{verbatim}
nobody:x:65534:65534:nobody:/var/empty:/bin/false
me:x:1000:1000::/home/me:/bin/bash
\end{verbatim}

\vskip 1cm

Pozn.: 3. sloupec je UID uživatele.

\end{frame}


\section{Příklad: Jednoduchý join}
\begin{frame}[fragile]{Příklad: Jednoduchý join}
\end{frame}

\end{document}


