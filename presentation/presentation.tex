\documentclass{beamer}
%
% Choose how your presentation looks.
%
% For more themes, color themes and font themes, see:
% http://deic.uab.es/~iblanes/beamer_gallery/index_by_theme.html
%
\mode<presentation>
{
  \usetheme{default}      % or try Darmstadt, Madrid, Warsaw, ...
  \usecolortheme{default} % or try albatross, beaver, crane, ...
  \usefonttheme{default}  % or try serif, structurebold, ...
  \setbeamertemplate{navigation symbols}{}
  \setbeamertemplate{caption}[numbered]
} 

\usepackage[czech]{babel}
\usepackage[utf8]{inputenc}
\usepackage[T1]{fontenc}
\usepackage{fontspec}

\def\icode#1{\texttt{#1}}


\title[Nástroj \icode{lsql-csv} na zpracování CSV souborů z příkazového řádku]{Nástroj \icode{lsql-csv} na zpracování CSV souborů z příkazového řádku}
\author{Přemysl Šťastný}
\institute{Univerzita Karlova}
\date{28.06.2024}

\begin{document}

\begin{frame}
  \titlepage
\end{frame}

% Uncomment these lines for an automatically generated outline.
%\begin{frame}{Outline}
%  \tableofcontents
%\end{frame}

\section{Úvod}

\begin{frame}{Úvod}

\begin{itemize}
  \item \icode{lsql-csv} je nástroj pro provádění krátkých dotazů nad daty malých CSV souborů z shellu
  \item Implementuje nový jazyk LSQL podobný SQL, speciálně navržený pro práci se CSV soubory v shellu
  \item Designovým cílem LSQL je být lapidárnějším jazykem než SQL. 
  \item Účelem jeho návrhu je umožnit uživateli rychle psát jednoduché dotazy přímo do terminálu.
\end{itemize}


\end{frame}


\section{Příklad: Hello World}

\begin{frame}[fragile]{Příklad: Hello World}

\begin{verbatim}
lsql-csv '-, &1.2 &1.1' <<- EOF
World,Hello
EOF
\end{verbatim}

\begin{verbatim}
Hello,World
\end{verbatim}

\end{frame}

\section{Příklad: Jednoduché filtrování}
\begin{frame}[fragile]{Příklad: Jednoduché filtrování}

\begin{verbatim}
lsql-csv -d: '-, &1.*, if &1.3 >= 1000' </etc/passwd
\end{verbatim}

\begin{verbatim}
nobody:x:65534:65534:nobody:/var/empty:/bin/false
me:x:1000:1000::/home/me:/bin/bash
\end{verbatim}

\vskip 1cm

Pozn.: 3. sloupec je \icode{UID} uživatele.

\end{frame}


\section{Příklad: Jednoduchý join}
\begin{frame}[fragile]{Příklad: Jednoduchý join}
\begin{verbatim}
lsql-csv -d: '/etc/{passwd,group}, &1.1 &2.1, 
  if &1.4 == &2.3'
\end{verbatim}

\begin{verbatim}
root:root
bin:bin
daemon:daemon
me:me
\end{verbatim}

\vskip 1cm

Pozn.: 4. sloupec \icode{/etc/passwd} je defaultní \icode{GID} uživatele a 3. sloupec \icode{/etc/group} je \icode{GID} skupiny.

\end{frame}


\section{Příklad: Jednoduché seskupení}
\begin{frame}[fragile]{Příklad: Jednoduché seskupení}
\begin{verbatim}
lsql-csv -d: 'p=/etc/passwd, p.7 count(p.3), by p.7'
\end{verbatim}

\begin{verbatim}
/bin/bash:7
/bin/false:7
/bin/sh:1
/bin/sync:1
/sbin/halt:1
/sbin/nologin:46
/sbin/shutdown:1
\end{verbatim}
\end{frame}


\section{Příklad: Jednoduché třídění}
\begin{frame}[fragile]{Příklad: Jednoduché třídění}
\begin{verbatim}
lsql-csv -d: '/etc/passwd, &1.*, 
  if &1.3 >= 1000, sort &1.3'
\end{verbatim}

\begin{verbatim}
me1:x:1000:1000::/home/me1:/bin/bash
me2:x:1001:1001::/home/me2:/bin/bash
me3:x:1002:1002::/home/me3:/bin/bash
nobody:x:65534:65534:nobody:/var/empty:/bin/false
\end{verbatim}
\end{frame}


\section{Příklad: Komplikovanější join}
\begin{frame}[fragile]{Příklad: Komplikovanější join}
\begin{verbatim}
lsql-csv -d: 'p=/etc/passwd g=/etc/group, 
  p.1 cat(g.1","), if p.1 in g.4, by p.1'
\end{verbatim}

\begin{verbatim}
adm:adm,disk,sys,
bin:bin,daemon,sys,
daemon:adm,bin,daemon,
lp:lp,
mythtv:audio,cdrom,tty,video,
news:news,
\end{verbatim}

\vskip 1cm

Pozn.: 1. sloupec \icode{/etc/passwd} je název uživatele, 1. sloupec \icode{/etc/group} je název skupiny a 
4. sloupec \icode{/etc/group} je seznam uživatelů skupiny.

\end{frame}

\section{Alternativní řešení}
\begin{frame}[fragile]{Alternativní řešení}
\begin{itemize}
\item SQL
\item Standardní linuxové nástroje
\item Other SQL on CSV implementations (\icode{q}, \icode{CSV SQL},\ldots)
\item General purpose language
\end{itemize}
\end{frame}
		

\section{Jak se \icode{lsql-csv} liší od SQL?}
\begin{frame}[fragile]{Jak se \icode{lsql-csv} liší od SQL?}
\begin{itemize}
\item lsql-csv dotazy jsou v mnoha usecasech lapidárnější než SQL dotazy
\item Data a jazyk SQL jsou typed. lsql-csv a CSV soubory jsou typeless.
\item Import CSV souboru do SQL databáze může být vhodnější řešení, 
  pokud jsou zapotřebí komplexnější queries, jsou zapotřebí joiny většího množství tables, 
  nebo je zapotřebí provést velké množství queries.
\end{itemize}
\end{frame}


\section{Jak se \icode{lsql-csv} liší od SQL?}
\begin{frame}[fragile]{Jak se \icode{lsql-csv} liší od SQL?}
\begin{verbatim}
SELECT dataX FROM data.txt WHERE dataX > 1000;
\end{verbatim}

\vskip 1cm

\begin{verbatim}
data.txt, &1.2, if &1.2 > 1000
\end{verbatim}

\end{frame}


\section{Rozdíly v použití \icode{lsql-csv} od standardních linuxových nástrojů při parsování CSV}
\begin{frame}[fragile]{Rozdíly v použití \icode{lsql-csv} od standardních linuxových nástrojů při parsování CSV}
\begin{itemize}
\item V některých případech dokáže být čitelnější a kratší
\end{itemize}
\end{frame}


\section{Rozdíly v použití \icode{lsql-csv} od standardních linuxových nástrojů při parsování CSV}
\begin{frame}[fragile]{Rozdíly v použití \icode{lsql-csv} od standardních linuxových nástrojů při parsování CSV}

\begin{verbatim}
lsql-csv -d: '-, &1.*, if &1.3 >= 1000' </etc/passwd
\end{verbatim}

\vskip 1cm

\begin{verbatim}
awk -F: '{ if($3 >= 1000){ print $0 }}' </etc/passwd
\end{verbatim}
\end{frame}


\section{Rozdíly v použití \icode{lsql-csv} od standardních linuxových nástrojů při parsování CSV}
\begin{frame}[fragile]{Rozdíly v použití \icode{lsql-csv} od standardních linuxových nástrojů při parsování CSV}
\begin{verbatim}
lsql-csv -d: '/etc/{passwd,group}, &1.1 &2.1, 
  if &1.4 == &2.3'
\end{verbatim}


\vskip 1cm

\begin{verbatim}
sort -t: -k3,3 /etc/group >/tmp/group.sort
sort -t: -k4,4 /etc/passwd >/tmp/passwd.sort

join -t: -14 -23 /tmp/passwd.sort /tmp/group.sort |
  cut -d: -f2,8
\end{verbatim}
\end{frame}


\section{Kde je balíček nahraný?}
\begin{frame}[fragile]{Kde je balíček nahraný?}
\begin{itemize}
    \item Hackage
    \item Stackage
    \item NixOS
\end{itemize}
\end{frame}

\end{document}


